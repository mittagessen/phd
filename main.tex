\documentclass[a4paper, 11pt]{book}

\RequirePackage[T1]{fontenc}
\usepackage{libertinus}

\usepackage{psl-cover}

\usepackage{subcaption}
\usepackage{graphicx} % Required for including images
\usepackage{tabularx}
\usepackage{footnote}
\usepackage[font=small,labelfont=bf]{caption} % Required for specifying captions to tables and figures
\usepackage{wrapfig} % Allows wrapping text around tables and figures
\usepackage{arydshln}
\usepackage{tikz}
\usepackage{pgfplots}
\usepackage{rotating}
\usepackage{flafter}

\usepackage{booktabs}
\usepackage{mathtools}
\usepackage{algorithm}
\usepackage{algorithmic}
\usepackage{gensymb}
\usepackage{epigraph}
\usepackage{csquotes}
\usepackage[toc,page]{appendix}
\usepackage{siunitx}
\usepackage{minted}
\setminted{fontsize=\footnotesize}
\setminted{breaklines}
\usepackage{etoolbox}
\AtBeginEnvironment{minted}{%
  \renewcommand{\fcolorbox}[4][]{#4}}
\usemintedstyle{bw}

\setmonofont{Inconsolata}
\usepackage{polyglossia}
\setdefaultlanguage{english}
\setotherlanguages{french,arabic}
\newfontfamily\arabicfont[Script=Arabic]{Amiri}

% nicer looking URLs
\usepackage{url}
\urlstyle{same}

\usepackage[style=ieee,backend=biber,maxnames=5]{biblatex}
\addbibresource{phd.bib}

\def\BibTeX{{\rm B\kern-.05em{\sc i\kern-.025em b}\kern-.08em
    T\kern-.1667em\lower.7ex\hbox{E}\kern-.125emX}}
\DeclarePairedDelimiter{\norm}{\vert}{\vert}

\makeatletter
\renewcommand\part{%
   \if@openright
     \cleardoublepage
   \else
     \clearpage
   \fi
   \thispagestyle{empty}%
   \if@twocolumn
     \onecolumn
     \@tempswatrue
   \else
     \@tempswafalse
   \fi
   \null\vfil
   \secdef\@part\@spart}
\makeatother

\usepackage{import}

\title{Avancées en Reconnaissance Optique des Caractères pour les Documents Arabes Historiques}

\author{Benjamin KIESSLING}

\institute{École Pratique des Hautes Études}
\doctoralschool{École doctorale de l'École Pratique des Hautes Études}{472}
\specialty{Voir spécialités ED}
\date{05 avril 2021}

\entitle{Advances in Optical Character Recognition for Historical Arabic Documents}

\jurymember{1}{Marc BUI}{Directeur d'études, École Pratique des Hautes Études}{Directeur de thèse}
\jurymember{2}{Nuria de CASTILLA}{Directeur d'études, École Pratique des Hautes Études}{Invité}
\jurymember{3}{Gregory CRANE}{Professor, Tufts University}{Rapporteur}
\jurymember{4}{Nachum DERSHOWITZ}{Professor, Tel Aviv University}{Rapporteur}
\jurymember{5}{Alicia FORNÉS}{Senior Research Fellow, Universitat Autònoma de Barcelona}{Examinateur}
\jurymember{6}{Daniel STÖKL BEN EZRA}{Directeur d'études, École Pratique des Hautes Études}{Examinateur}
\jurymember{7}{Peter STOKES}{Directeur d'études, École Pratique des Hautes Études}{Examinateur}

\frabstract{
La transcription automatique de textes dans les documents historiques
manuscrits et imprimés est devenue un processus établi dans les humanités
numériques, son utilisation allant des archives ou des bibliothèques à grande
échelle aux groupes de recherche et aux chercheurs individuels. Bien que des
progrès considérables aient été réalisés ces dernières années pour comprendre
les limites et faire progresser l'état de l'art, ces recherches restent
largement limitées aux documents écrits dans les systèmes d'écriture européens,
et plus particulièrement à l'écriture latine. L'une des cultures littéraires
les plus vastes et les plus diverses, largement ignorée par les recherches
actuelles sur l'analyse d'images de documents, est l'écriture arabe.

Cette thèse contient une étude complète sur les caractéristiques des documents
en écriture arabe et les défis qu'ils posent aux systèmes de reconnaissance
optique de caractères de pointe, à travers une analyse théorique de l'écriture
arabe et deux études de cas de rétro-numérisation sur des documents imprimés
classiques et modernes. Les principales limites des méthodes courantes
identifiées dans ces études ont ensuite été traitées. Deux méthodes
entraînables de segmentation des pages suivant le paradigme de la ligne de
base, permettant d'obtenir des résultats comparables à l'état de l'art et
comprenant des caractéristiques supplémentaires nécessaires à la segmentation
de pages de documents complexes, une méthode simple de traitement des lignes de
texte multigraphiques et le logiciel ROC flexible Kraken intégrant ces méthodes
sont présentés. On montre l'utilité de ce logiciel de ROC non seulement pour la
reconnaissance de texte traditionnelle mais aussi pour une nouvelle tâche
d'alignement des caractères. En outre, on présente l'environnement de recherche
virtuel (ERV) eScriptorium pour l'annotation et la transcription. Cet ERV est
spécifiquement conçu pour pouvoir traiter des textes non-latins, dont l'arabe,
plus efficacement que les systèmes alternatifs existants. Au cours de ce
travail, on a également préparé plusieurs ensembles de données d'entraînement
et d'évaluation sous licence ouverte pour la transcription de textes arabes et
la segmentation de pages.
}

\enabstract{
The automatic transcription of text in handwritten and machine-printed
historical documents has become an established process in the Digital
Humanities, its use ranging from large scale archival or library settings to
research groups and individual scholars. While considerable progress on
understanding limitations and advancing the state of the art has been made in
recent years, this research remains largely limited to documents written in
European writing systems, most importantly the Latin script. One of the largest
and most diverse literary cultures largely ignored by current document image
analysis research is the Arabic one.

This thesis contains a comprehensive study on the features of Arabic-script
documents and their challenges posed to state of the art optical character
systems through both a theoretical analysis of the Arabic script and two case
studies of retrodigitization on printed classical and modern material. The
principal limitations of common methods identified in these studies were
subsequently addressed. Two trainable layout analysis methods following the
baseline paradigm achieving comparable results to the state of the art while
incorporating additional features necessary for the segmentation of complex
document pages, a basic method for processing of multigraphic text lines, and
the flexible Kraken OCR engine integrating these methods are presented. We show
the usefulness of this OCR software not only for traditional text recognition
but also a novel character alignment task. Further, we present the eScriptorium
virtual research environment (VRE) for annotation and transcription. This VRE
is specifically designed to be able to treat non-Latin, among them Arabic,
script material more effectively than existing alternative systems. In the
course of this work we also prepared multiple openly licensed training and
evaluation datasets for Arabic text transcription and layout analysis.
}

\frkeywords{segmentation des pages, reconnaissance de texte, environnement de recherche virtuel, écriture arabe, analyse d'images de documents}
\enkeywords{layout analysis, text recognition, virtual research environments, arabic, document image analyis}

\begin{document}

\pslcover

\frontmatter
\chapter*{Acknowledgements}
\thispagestyle{empty}

First, I would like to express my deepest gratitude to Marc Bui, who
accepted to supervise this thesis, for very interesting discussions, for his
valuable experience and guidance in matters both scientific and administrative.

I am so grateful to the eScripta team at EPHE and everyone at the Chair for
Digital Humanities at the University of Leipzig. Both provided an environment
were I was free to explore research ideas and their generous financial support
made my existence as a doctoral student a comfortable one. It was Maxim Romanov
through whom I had the first contact with Arabic text in Leipzig. I am also
eternally indebted to Daniel Stökl Ben Ezra and Peter Stokes for fending off
every imposition on my time during the last weeks of writing. 

I thank the OpenITI team for their incessant lobbying to get institutional
support for research into Arabic OCR. Without them, none of this would have
been possible. A special thanks goes to Matthew Thomas Miller and Jonathan
Allen who responded to my endless question on details of the Arabic script with
a celerity unheard of in academia.

The people of ALMAnaCH at INRIA should not be forgotten. They welcomed a
clueless foreigner on his arrival in France and integrated me in the lab.

I have wonderful friends who were there even at times when all they heard from
me was about this thesis. A huge thanks goes to Louise for helping improve the
French summary. Above all I thank Monika for reminding me in the darkest
moments that instant ramen is not the foundation of a balanced diet.

\cleardoublepage
\tableofcontents
\listoffigures
\listoftables

\mainmatter
\import{./chapters}{intro.tex}
\cleardoublepage
\part{The Arabic Writing System}
\import{./chapters}{arabic.tex}
\cleardoublepage
\begin{refsection}
\import{./chapters/alwusta}{soa.tex}
\printbibliography[heading=subbibliography,title={Editions of Printed Texts},type=book]
\printbibliography[heading=subbibliography,nottype=book,notkeyword=soa]
\end{refsection}
\cleardoublepage
\begin{refsection}
\import{./chapters/champs}{advances.tex}
\printbibliography[heading=subbibliography,notkeyword=champs]
\end{refsection}
\cleardoublepage
\part{Layout Analysis and Segmentation}

\label{part:la}
\begin{refsection}
\import{./chapters/hip}{icdar}
\printbibliography[heading=subbibliography,notkeyword=hip]
\end{refsection}
\cleardoublepage
\begin{refsection}
\import{./chapters/icfhr2020}{seg}
\printbibliography[heading=subbibliography,notkeyword=icfhr]
\end{refsection}
\cleardoublepage
\begin{refsection}
\import{./chapters/read}{read}
\printbibliography[heading=subbibliography,notkeyword=read]
\end{refsection}
\cleardoublepage

\part{Transcription and Character Alignment}

\begin{refsection}
\import{./chapters/dh2019}{dh}
\printbibliography[heading=subbibliography,notkeyword=dh]
\end{refsection}
\cleardoublepage
\begin{refsection}
\import{./chapters/icfhr2020algn}{Transcribe}
\printbibliography[heading=subbibliography,notkeyword=algn]
\end{refsection}
\cleardoublepage

\part{The Escriptorium VRE}

\begin{refsection}
\import{./chapters/ost}{ost}
\printbibliography[heading=subbibliography,notkeyword=ost]
\end{refsection}
\cleardoublepage
\begin{refsection}
\import{./chapters/manuvre}{vre}
\printbibliography[heading=subbibliography,notkeyword=vre]
\end{refsection}
\cleardoublepage
\import{./chapters/}{conclusion}
\cleardoublepage
\begin{appendices}
\chapter{Résumé Long}

\begin{french}

\section{Introduction}

Cette thèse consiste en un certain nombre de publications qui étudient les
difficultés de la rétrodigitalisation de l'écriture arabe historique et propose
plusieurs méthodes pour faire progresser l'état de l'art en la matière. Bien
que ces méthodes visent principalement à remédier les problèmes résultant des
caractéristiques de l'écriture arabe, elles sont conçues pour être applicables
à l'écriture et aux inscriptions dans une variété d'autres systèmes d'écriture.

\section{L'analyse d'imges de documents et reconnaissance optique de caractères}

L'analyse d'images de documents (AID) est un sous-domaine de la vision par
ordinateur (VO) qui cherche à comprendre le contenu des documents par
le traitement de leurs images numériques.  Les documents sont définis
de façon libérale par le domaine, comprenant pas seulement les
documents manuscrits et le texte imprimé sur papier, mais aussi
l'écriture sur d'autres supports souples tels que le papyrus ou les
feuilles de palmier et même les inscriptions.

La différence vis-à-vis de la vision par ordinateur en général ne se trouve pas
dans les méthodes employées, mais dans la nature des images traitées.
Ces images sont généralement obtenues par des caméras ou des scanners,
souvent dans un contexte professionnel, ce qui permet d'obtenir un
matériel source avec un minimum de bruit provenant d'éléments non
pertinents que l'on rencontre souvent dans les images de scènes
naturelles traitées par d'autres branches du VO. Malgré des données d'entrée
plus propres, les représentations structurées souhaitées en sortie ont
tendance à être plus complexes et plus nombreuses dans la AID que dans
d'autres applications, nécessitant la détection, la classification et
la mise en relation de dizaines d'éléments de documents tels que des
lignes, des caractères, des illustrations et des tableaux.

Comme dans d'autres domaines de l'informatique, la recherche en DIA peut être
divisée en tâches spécifiques, dont une ou plusieurs sont résolues par
une méthode particulière proposée. La tâche la plus importante de la
DIA est la reconnaissance optique de caractères (ROC), mais d'autres,
basées sur la ROC ou entièrement nouvelles, telles que la
reconnaissance de documents de classification, de datation ou de
repérage par mot clé, existent. La ROC, la conversion de textes imprimés,
écrits ou inscrits en texte codé par machine, est un processus établi depuis
longtemps, à la fois en tant que tâche dans la recherche sur la vision
par ordinateur et son utilisation pratique dans des applications comme
l'analyse des adresses ou les aides aux aveugles. Il est sans doute à
l'origine de toute AID avec un certain nombre de brevets remontant
au début du XIXe siècle.

Ces premières approches, des décennies avant les premiers ordinateurs, ont
maintenant évolué en l'application quotidienne des techniques de DIA
pour des tâches telles que l'analyse des adresses pour la distribution
du courrier, la vérification des chèques et la rétro-numérisation des
livres. En effet, on revendique aujourd'hui dans la profession que la
ROC est fondamentalement résolue, au moins pour les documents imprimés
modernes en anglais avec un niveau de bruit raisonnablement faible, où
les logiciels commerciaux modernes de rétrodigitalisation atteignent
régulièrement des taux de précision des caractères supérieurs à 99\%.
Néanmoins, il existe près de 4000 langues écrites et plusieurs centaines de
systèmes d'écriture, pour la grande majorité desquels des systèmes de
ROC pratiques ne sont pas disponibles. Même en considérant seulement
l'utilisation d'écritures purement alphabétiques telles que le latin et
le cyrillique, qui posent moins de problèmes à la ROC de pointe
lorsqu'elles sont employées conformément aux pratiques typographiques
occidentales modernes, il est clair qu'une part importante de la
production littéraire humaine n'est pas encore accessible par la
rétro-numérisation.

C'est encore plus clair pour le matériel historique. Les projets de
numérisation à grande échelle dans les pays riches ont permis de créer de
vastes collections numériques de les bibliothèques qui sont de facto
inaccessibles au public et aux chercheurs, même pour des documents aussi
récents que ceux de la fin du XIXe siècle car les variations orthographiques et
typographiques dégradent considérablement la qualité du texte numérisé par les
logiciels adaptés aux documents modernes. Il s'agit très probablement d'un état
de fait temporaire pour les documents les plus nombreux dans les archives des
pays développés où des projets tels que OCR-D\footnote{\url{http://ocr-d.de}}
ouvrent la voie pour transférer les progrès de la recherche pure en AID à la
pratique des bibliothèques. Dans le cadre de ces projets et d'efforts plus
spécifiques tels que \cite{smith2018research}, on a vu la cristallisation d'un
programme de recherche collective pour la numérisation du matériel historique
et des systèmes d'écriture minoritaires, partagé par les chercheurs en sciences
humaines engagés dans les méthodes numériques et les experts de la vision par
ordinateur. Néanmoins, ces communautés restent fracturées sur le plan
géographique, les frontières linguistiques et professionnelles.

D'autre part, le manque de financement combiné à la détérioration des
collections dans les autres pays en raison de conflits et d'influences
environnementales augmente le risque de perte permanente du patrimoine culturel
qui n'intéresse qu'un petit nombre de chercheurs et de populations
minoritaires. Même des collections célèbres telles que les manuscrits de
Tombouctou ont à peine échappé à la destruction par les conflits de ces
dernières années et sont en grand danger d'être détruites par l'humidité. 

\subsection{Tâches}

En tant qu'application centrale de l'analyse d'images de documents, la ROC
s'est divisée en un grand nombre de sous-problèmes. Ils ne sont pas tous
strictement nécessaires à un système ROC fonctionnel et, en fait, beaucoup
d'entre eux ne peuvent être mis en œuvre que de manière spécifique aux
matériaux et sont donc relégués à des applications spécialisées.

Le pipeline typique de la reconnaissance optique de caractères est construit
autour de quatre étapes de traitement de base :

\begin{description}
\item [Prétraitement] Le débruitage, le redressement, et la binarisation
\item [Segmentation des pages] Extraction des informations structurelles des
images de pages de documents et les enrichir avec des informations sémantiques
supplémentaires.
\item [Transcription] Extraction d'informations textuelles de la totalité ou
d'un sous-ensemble des objets identifiés à l'étape précédente.
\end{description}

Si cette caractérisation est valable pour tous les pipelines de données, sauf
les plus ésotériques, les blocs fonctionnels exacts dépendent fortement du type
et de la structure des documents à traiter. Chacune de ces étapes de traitement
contient une ou plusieurs méthodes qui servent à résoudre une tâche
particulière, comme par exemple:

\begin{description}
\item [Binarisation] La classification des pixels d'une image en deux classes :
le front, c'est-à-dire le texte, et le fond, c'est-à-dire tout le
reste.
\item [Débruitage] Augmenter la qualité de l'image de la page pour les tâches
	suivantes. Le débruitage comprend des processus tels que la
		normalisation de fond ou l'élimination des taches.
\item [Deskewing and Dewarping] Corriger à la fois la distorsion de perspective
	inhérente à la capture par caméra et d'autres dégradations introduites
couramment pendant de scanning telles que la
rotation, le déformation le long de la reliure, \dots
\item [Segmentation en regions]  Subdiviser une image de page en éléments tels que du texte, de la décoration, des notes, des points, etc.
\item [Segmentation en lignes] Extraction des lignes de texte d'une image de page.
\item [Segmentation en caractères] Segmentation du texte sur une image de page
	jusqu'au niveau du glyphe ou même à un niveau inférieur. Bien qu'il
		s'agisse d'une opération courante dans les systèmes ROC
		traditionnels, elle est le plus souvent superflue avec les
		méthodes de pointe.
\item [Classification du système d'écriture et de la police de caractères] Classifier la langue, le système d'écriture, le style ou la police de
caractères du texte. Cette classification peut être effectuée à
différents niveaux, par exemple à l'échelle du document ou
individuellement pour tout ou partie d'une ligne.
\item [Reconnaissance des tableaux] Inférer la structure logique des images des tableaux.
\end{description}

\section{Motivation et contributions scientifiques}

L'écriture arabe représente l'une des plus grandes traditions littéraires de
l'histoire de l'humanité, tant en termes de volume que de diffusion
géographique. Les exemples vont des textes religieux, en particulier le Coran,
le livre saint de l'Islam, à la poésie, en passant par les textes scientifiques
et juridiques, sans oublier un vaste corpus de documents administratifs. Le
nombre considérable de ces textes dans une multitude de domaines en fait une
cible privilégiée pour les nouveaux paradigmes des sciences humaines qui
utilisent des méthodes computationnelles telles que la lecture à distance et la
paléographie quantitative. Ces méthodes nécessitent soit de grands corpus de
textes, soit des méthodes précises de AID basées sur une ou plusieurs des
tâches susmentionnées d'un système ROC. Comme la grande majorité des textes
arabes n'ont jamais existé sous forme numérique, la rétro-numérisation de haute
qualité par ROC constitue la base d'un nombre important de projets de recherche
en sciences humaines numériques arabes.

Lorsque j'ai commencé à travailler sur cette thèse, la ROC du texte arabe
imprimé a été largement rejetée par les chercheurs en sciences humaines
travaillant sur ces documents, même ceux qui étaient déjà très impliqués dans
les sciences humaines numériques, et par les organismes de financement comme
peu pratique, encore plus pour le matériel historique ou multigraphique. La
reconnaissance précise de l'écriture à la main arabe semblait totalement
inatteignable. Bien qu'il existe une longue tradition de publications sur la
ROC d'ensembles de données arabes simples tels que KHATT\cite{mahmoud2014khatt}
et qu'un certain nombre de logiciels de ROC ouverts et propriétaires tels que
Tesseract, Abbyy FineReader et Sakhr offrent un soutien nominal pour la
reconnaissance de textes en caractères arabes, ces solutions ne se sont jamais
concrétisées dans la pratique réelle des bibliothèques universitaires ou à
grande échelle. Les raisons en sont multiples : taux d'erreur élevé des
classificateurs et des segmenteurs mal adaptés à la nature cursive du système
d'écriture, manque de logiciels et de compétences techniques facilement
disponibles, et coûts et efforts substantiels nécessaires pour adapter les
solutions existantes au matériel d'intérêt.

Il est rapidement apparu que les obstacles qui empêchent les chercheurs
travaillant sur des textes imprimés et manuscrits en caractères arabes
d'utiliser la ROC dans la pratique sont les mêmes que ceux rencontrés par de
nombreux autres chercheurs engagés dans la rétro-numérisation de documents
historiques et non-latins. Imitant l'opinion dominante sur la ROC arabe,
\cite{widner2017toward} a revendiqué que les manuscrits médiévaux (en
caractères latins) étaient pratiquement imperméables à la ROC contemporaine. On
peut probablement trouver des évaluations similaires pour d'autres domaines.

En tant que telle, cette thèse doit être considérée à l'intersection plus
générale entre les sciences humaines et l'informatique. La recherche qui y est
présentée n'est pas un ensemble de méthodes non connectées permettant de
résoudre des tâches uniques dans l'AID arabe, mais fait partie d'un écosystème
cohérent pour l'AID des sciences humaines composé de deux éléments principaux :
le système ROC Kraken et l'environnement de recherche virtuel (ERV)
eScriptorium.

Le Kraken est un système ROC complet et modulaire. Il se distingue des autres
solutions de plusieurs façons importantes : les chercheurs en sciences humaines
à tendance numérique comme utilisateurs et la conception du logiciel qui
cherche une flexibilité maximale. Il a également été largement adapté en vue
d'un meilleur traitement des documents historiques en caractères non latins,
notamment en arabe.

Dans le cadre des travaux d'adaptation visant à faire de Kraken un outil plus
performant pour le travail sur les textes arabes, deux études de faisabilité de
la rétro-numérisation de livres imprimés en caractères arabes classiques et
d'une importante revue de langue arabe publiée par l'Université Américaine de
Beyrouth ont été réalisées, qui ont produit la première analyse détaillée sur
les faiblesses et les capacités des méthodes de ROC de pointe sur les textes
arabes imprimés.

Cette thèse contribue de deux systèmes de segmentation de lignes entraînables,
un système élémentaire capable de détecter uniquement les lignes de base, et un
système plus avancé permettant la segmentation des régions et des lignes en
plus de la classification. Ce dernier est inclus dans le Kraken et permet la
détection conjointe de régions et de lignes et l'inférence de l'orientation des
lignes. Cette seconde méthode a également été optimisée pour l'utilisation de
la mémoire, réduisant la consommation de mémoire d'environ cinquante pour cent
par rapport aux réseaux de neurones à segmentation sémantique à convolution
totale, ayant des performances similaires.

Un nouveau backend de réseau neuronal a été ajouté au Kraken. Basé sur la
bibliothèque de réseaux de neurones pytorch, il permet la reconfiguration
flexible des réseaux de neurones artificiels (ARN) utilisés à la fois pour la
segmentation des pages et la transcription du texte grâce à un langage de
définition d'ARN léger qui est capable d'exprimer de nombreuses
caractéristiques des architectures courantes utilisées dans les tâches de
vision par ordinateur. Cette couche permet l'ajout relativement simple de
nouveaux types d'ARN et donc un prototypage rapide et une optimisation efficace
des hyperparamètres, même pour les utilisateurs finaux sans connaissances
approfondies en matière d'apprentissage machine, comme l'a démontré
\cite{strobel2020much}.

Au cours des premières études de cas, plusieurs milliers de lignes de données
de formation pour la transcription de textes ont été annotées et mises à la
disposition du public. J'ai participé à la conceptualisation technique et aux
directives de transcription pour ces ensembles de données. Un autre ensemble de
données sous licence libre de quatre cents pages de manuscrits en écriture
arabe dans une variété de langues, de styles et de domaines a été annoté avec
des lignes de base et l'orientation des lignes pour permettre l'évaluation des
méthodes d'analyse de la mise en page sur des documents historiques en écriture
arabe.  Cet ensemble de données très complexe reste le seul ensemble de données
manuscrites non latines pour le paradigme de base de l'analyse de la mise en
page.

La deuxième composante de cet écosystème de rétro-numérisation est l'ERV
eScriptorium.  Alors que le Kraken est conçu pour une flexibilité maximale,
l'eScriptorium adopte une autre approche. Conçu comme un environnement de
recherche paléographique et de publication à part entière pour un usage
scientifique, la fonctionnalité ROC n'est qu'une petite partie des
caractéristiques prévues. Comme il n'est pas pratique d'exposer toutes les
fonctionnalités du Kraken sur une telle plate-forme, la conception
d'eScriptorium permet une intervention manuelle et semi-automatique à chaque
étape du processus, soit par une manipulation manuelle dans l'interface, soit
par des interfaces d'échange de données graphiques et programmatiques.

Comme eScriptorium vise à offrir des fonctions scientifiques supplémentaires
au-delà de la simple rétro-numérisation, la plateforme est également un cadre
de test idéal pour la recherche en sciences humaines assistée par la vision
artificielle. Dans certains cas, ces fonctions avancées sont liées à la
transcription de textes. Une méthode permettant de dériver les emplacements des
graphèmes à partir de l'alignement implicite produit par une reconnaissance de
texte par ligne ANN et une évaluation sur des fragments de matériel hébraïque
est présentée.

\section{L'écriture arabe}

Le système d'écriture arabe est l'un des systèmes d'écriture les plus répandus
de l'histoire de l'humanité, géographiquement et chronologiquement. Il s'agit
de l'écriture principale de la langue arabe, du farsi, de l'ourdou et de
plusieurs autres langues du sous-continent indien. Historiquement, elle a été
utilisée pour produire des textes de l'espagnol au chinois.

Bien que ses origines exactes soient encore controversées, les historiens
s'accordent à dire que l'écriture arabe a évolué à partir de l'écriture
nabatéenne ou syriaque au Moyen-Orient au cours de plusieurs siècles, la
maturation ayant eu lieu au cours du septième siècle de notre ère.  Étroitement
liés à la propagation de l'Islam, un certain nombre de variantes alphabétiques
et de styles calligraphiques régionaux se sont développés au cours des siècles
suivants. Néanmoins, on est loin d'une écriture purement liturgique avec une
abondance de documents administratifs, de traités philosophiques et
scientifiques, de poésie, etc.

Il est donc erroné de parler d'une seule écriture arabe du point de vue des
recherches de l'AID. Chaque style, en combinaison avec les préférences
régionales et le contexte de leur utilisation, présente des défis particuliers.

\subsection{Les principes de l'écriture arabe}

L'écriture arabe est un \emph{abjad}, un système d'écriture consonantique, ce qui
signifie qu'elle ne nécessite que l'écriture des consonnes et des voyelles
longues, le lecteur étant censé fournir lui-même les voyelles courtes
appropriées selon le contexte.  Les voyelles courtes et autres marques pour des
caractéristiques telles que le doublage (gémination) et la nunation (ajout d'un
n final), peuvent être ajoutées en option (\emph{tashkil}) mais ne sont
systématiquement utilisées que lors de la transcription du Coran ou de textes
élémentaires pour les apprenants en langues. Comme le syriaque et l'hébreu, il
s'écrit de droite à gauche, à l'exception des nombres qui s'écrivent de gauche
à droite.

Contrairement à d'autres écritures très courantes comme le latin, l'arabe est
écrit uniquement dans une forme cursive. Comme les variantes cursives d'autres
écritures, les lettres individuelles changent de forme en fonction de leur
position dans un mot : des formes initiales, médianes, finales et indépendantes
existent, en plus des règles calligraphiques spécifiques au style pour le
placement des lettres. Une différence avec l'écriture cursive des autres
écritures est que toutes les lettres ne peuvent pas être reliées entre elles.
Comme les espaces n'indiquent pas nécessairement le début d'un nouveau mot, les
calligraphes sont largement libres de faire varier l'espacement entre les mots,
les syllabes et les coups comme ils le souhaitent.  Cette variation de
l'espacement peut perturber les systèmes de reconnaissance optique des
caractères.

Une autre distinction est l'absence de majuscule et la ponctuation de type
occidental, cette dernière n'ayant été introduite qu'au XXe siècle de notre
ère. Au lieu de cela, des mots et des phrases particuliers sont utilisés pour
introduire une nouvelle phrase ou question et les accents sont placés au-dessus
des titres et des rubriques.

Il existe d'autres formes de lettres dans des variantes de l'écriture arabe
adaptées à d'autres langues qui comportent des phonèmes qui ne se trouvent pas
en arabe. Elles sont généralement adaptées en ajoutant des points aux graphèmes
représentant des sons similaires en arabe, comme le persan \emph{pe} dérivé de
\emph{bāʾ} en ajoutant deux points supplémentaires ci-dessous. Comme il y a
beaucoup de langues qui sont ou ont été écrites dans l'écriture, il existe un
grand nombre de ces variantes.

Une difficulté particulière pour les systèmes ROC est la façon particulière
dont le texte arabe est justifié. La césure, c'est-à-dire la division des mots
pour faciliter le retour à la ligne, est absente de tous les textes, sauf les
plus anciens. Au lieu de la justification par espace blanc, courante dans les
écritures alphabétiques, des alternatives plus agréables visuellement ont été
conçues.  L'allongement des liens entre les lettres, la courbure de la ligne de
base et la superposition ou le déplacement de fragments du dernier mot
au-dessus ou à côté de la ligne sont des phénomènes courants, ces deux
dernières méthodes posant un défi particulier, même aux systèmes modernes de
segmentation des pages.

\subsection{Styles}

Un grand nombre de styles calligraphiques ont été conçus au cours des siècles.
Si certains sont reconnus dans l'ensemble du monde islamique, comme le
\emph{naskh}, d'autres sont spécifiques à certaines zones géographiques, par
exemple le \emph{maghribi} nord-africain ou le \emph{nastaʼlīq} persan. Les
premiers styles tels que le \emph{ḥijāzī} et le coufique sont des familles de
styles différents, car la standardisation était faible. Au treizième siècle de
notre ère, six styles canoniques étaient apparus. Ils sont généralement
appariés, une écriture d'affichage (majuscule) et une écriture de texte
(minuscule) :

\begin{itemize}
        \item \emph{ṯuluṯ} et \emph{naskh}
        \item \emph{muḥaqqaq} et \emph{rayḥān}
        \item \emph{tawqīʿ} et \emph{riqāʿ}
\end{itemize}

Les styles régionaux n'étaient pas seulement liés aux zones géographiques mais
aussi à l'utilisation de la langue.  Dans les parties du monde islamique
influencées par le persan (Empire Ottoman, Iran, Inde), des styles suspendus
tels que \emph{nastaʼlīq} avec des mots individuels descendant sur une ligne de
base commune étaient populaires car ils étaient plus adaptés aux différentes
combinaisons de lettres du turc et du persan. D'autres styles étaient à la fois
géographiquement limités et restreints à certains usages comme le
\emph{divanî} style chancellerie ottomane.

\subsection{Critères pour les systèmes de ROC arabes}

En résumé, la reconnaissance des textes arabes imprimés et manuscrits nécessite
un certain nombre de caractéristiques que l'on ne trouve pas couramment dans
les systèmes ROC actuels. Les principales exigences sont, dans l'ordre de
traitement à l'intérieur d'un pipeline typique :

\begin{description}
	\item[Èlimination de la binarisation] La variété des supports, des encres et des décorations utilisés
		dans les textes arabes rend peu probable le développement d'une
		méthode générale de binarisation. Une système ROC arabe devrait
		donc être sans binarisation.
	\item[Segmentation des lignes courbes et inclinées] L'utilisation
		fréquente de lignes inclinées et courbes à des fins tant
		pratiques qu'esthétiques nécessite une méthode de segmentation
		des pages capable de les extraire et de les représenter
		efficacement.
	\item[Segmentation sémantique des pages] L'utilisation extensive du
		paratexte nécessite un système de segmentation capable de
		séparer plusieurs textes sur la même page de document.
	\item[Détermination de l'ordre de lecture avancée] En plus de
		classifier les différents éléments textuels d'une page, il est
		également nécessaire de les mettre dans le bon ordre pour une
		véritable compréhension du document.
	\item[Transcription sans segmentation] Comme l'arabe s'écrit uniquement
		sous forme cursive et que les liaisons entre les lettres
		peuvent changer de manière significative d'un style à l'autre,
		une méthode de transcription arabe devrait fonctionner sur des
		lignes entières au lieu de les séparer en caractères.
	\item[Les outils de création et de conservation des données]
		Même s'ils ne font pas directement partie d'un pipeline de ROC,
		les outils ergonomiques qui peuvent être utilisés pour annoter
		toute la gamme des caractéristiques du texte arabe sont
		essentiels pour les projets de numérisation concrets et pour
		créer des ensembles de données pour les recherches futures.
\end{description}

\section{Études sur le ROC en écriture arabe}

Deux études de précision sur un certain nombre de textes classiques en écriture
arabe et sur la principale revue savante de langue arabe, al-Abhath, ont été
réalisées avec une version antérieure du système ROC Kraken afin de déterminer
les points forts et les limites des logiciels ROC actuels sur ces documents. En
conséquence, nous avons déterminé deux recommandations clés pour améliorer
substantiellement la ROC en écriture arabe : (1) une approche plus systématique
de la production de données de l'entrainement, et (2) le développement de
composants technologiques clés, en particulier des modèles multilingues et une
meilleure méthode de segmentation des lignes et de la mise en page.

La première étude préliminaire a montré que malgré les faibles taux d'erreur de
caractères (<3\%) obtenus par les modèles spécifiques aux polices de
caractères, les différences substantielles entre les polices de caractères se
traduisent par des taux d'erreur nettement plus élevés desdits modèles sur des
polices de caractères dissemblables. Bien que cette mauvaise généralisation
puisse être attribuée dans une certaine mesure aux limites du réseau de
transcription (LSTM peu profond entraîné sans régularisation) dans cette
première version du Kraken, elle a montré la nécessité à la fois d'un backend
de réseau neuronal plus puissant dans le système et d'une sélection minutieuse
des données d'entraînement pour assurer la représentativité de l'ensemble des
données d'entraînement pour le domaine cible souhaité.

La seconde étude a été construite dès le début autour d'une analyse beaucoup
plus rigoureuse des polices de caractères présentes dans le matériel et d'une
évaluation manuelle approfondie des résultats de la ROC par rapport à une
évaluation purement computationnelle. Son sujet, la revue al-Abhath, a été
déterminée à comporter deux groupes de polices de caractères, bien qu'il y ait
de légères variations intra-groupe, en particulier pour la première police de
caractères qui a été utilisée pour la majorité du cycle de la revue. Pour se
conformer à cette analyse, il a été décidé de produire 5000 lignes de données
d'entraînement pour la première police de caractères et 2000 pour la seconde.
Les lignes d'entraînement ont été tirées au hasard des deux groupes de
caractères et transcrites manuellement avec le logiciel CorpusBuilder. Des
modèles de transcription spécifiques aux polices de caractères ont été
entraînés sur les deux ensembles de données, puis évalués par un prestataire
extérieur et OpenITI sur un ensemble de validation distinct (voir tableaux
~\ref{tab_champs:table_1} et ~\ref{tab_champs:table_2}). À l'issue de l'examen,
il est apparu clairement que Kraken surpasse significativement le logiciel ROC
d'Abbyy pour la reconnaissance de textes arabes. Une évaluation manuelle
détaillée de la transcription automatique a permis d'identifier plusieurs types
d'erreurs courantes : mauvaise reconnaissance de ligatures peu courantes, de
formes de lettres rares et de caractères d'allongement, de texte multigraphique
et le sortie de lettres doublées. Bien que certains de ces problèmes aient pu
être résolus grâce à des données d'entraînement de meilleure qualité, le
traitement de texte multigraphique dans le système ROC a été ajouté
subséquemment (chapitre~\ref{ch:multi}). L'évaluation a également identifié le
besoin d'une meilleure segmentation des lignes des textes arabes, car le module
optimisé pour le latin dans le Kraken avait tendance à tronquer et à diviser
les lignes de texte avec une certaine fréquence.

\section{Segmentation des pages}

Pour le traitement des documents imprimés et manuscrits en caractères arabes,
il est évident qu'une méthode d'analyse de la mise en page flexible, et de
préférence entraînable, est nécessaire, vu que l'impossibilité d'extraire des
lignes de texte rend automatiquement impossible leur transcription correcte
avec n'importe quelle méthode de reconnaissance de texte.

Alors qu'il existe un grand nombre d'ensembles de données d'entraînement
suivant différentes représentations de lignes de texte pour les documents
modernes et historiques en écriture latine, ce n'est pas le cas pour les
manuscrits arabes. Pour aider au développement, nous avons préparé un ensemble
de données de 400 pages de manuscrits arabes et persans annotés avec leurs
lignes de texte.

Il est important de comprendre la nature exacte de la segmentation des lignes
de texte dans un pipeline ROC et son lien avec la méthode de transcription. Si
l'objectif premier est d'identifier les lignes de texte, la tâche d'un
segmenteur de lignes de texte est également d'aider à l'extraction des lignes
de manière à optimiser les performances de la transcription. Ainsi, différentes
représentations de lignes de texte peuvent être produites par un système de
segmentation, par exemple des boîtes englobantes, des polygones englobants, des
nuages de pixels, des lignes de base, etc. Tous ces éléments ne sont pas
adaptés aux conventions calligraphiques de l'écriture arabe. Par exemple, le
tracé d'un rectangle englobant une ligne courbe ou inclinée inclura
nécessairement des parties de lignes adjacentes à l'intérieur du rectangle.  En
outre, la précision de la transcription est améliorée lorsque les lignes sont
centrées à l'intérieur de la bande d'image rectangulaire introduite dans le
réseau de transcription.

Une représentation capable à la fois d'encoder une ligne sans bruit adjacent et
de permettre la normalisation de la ligne sur une ligne droite est donc
hautement souhaitable pour les documents qui contiennent des lignes non droites
avec une certaine régularité.  Pour l'ensemble de données, la représentation de
la ligne de base a été choisie car elle est à la fois rapide à annoter,
relativement facile à apprendre par les méthodes de vision par ordinateur, et
suffisamment polyvalente pour permettre la normalisation de lignes de forme
arbitraire.

Une ligne de base n'est qu'une ligne virtuelle sur laquelle la plupart des
caractères reposent. Alors qu'ils soient généralement droits dans les documents
imprimés, ils peuvent être définis comme des polylignes, ce qui leur permet de
suivre la courbure de la ligne. En projetant les éléments courbes sur une ligne
de base droite, nous pouvons transformer une ligne courbe en une ligne droite
pouvant être traitée par le modèle de transcription. Si la ligne de base est
associée à un polygone englobant, il est également possible de supprimer des
éléments en dehors de la ligne de texte qui nous intéresse dans le processus.

Nous avons proposé une méthode de segmentation, basée sur un réseau neuronal de
segmentation sémantique convolutive profonde (U-Net), suivant cette
représentation de la ligne de base et l'avons évaluée sur notre ensemble de
données et un ensemble de données latins cBAD. La méthode a atteint une
précision similaire à celle d'autres méthodes de pointe (voir
tableau~\ref{tab:foo}).   Bien que les résultats obtenus sur l'ensemble de
données arabes aient été inférieurs à ceux obtenus sur l'écriture latine, notre
ensemble de données est beaucoup plus diversifié, ce qui indique la pertinence
générale de cette approche pour la segmentation des lignes de texte des
manuscrits arabes.

Bien qu'efficace, cette méthode manquait de plusieurs caractéristiques
nécessaires à un segmenteur pratique. Premièrement, elle ne comporte pas de
moyen de calculer l'orientation des lignes, deuxièmement, elle ne dispose pas
d'un algorithme pour calculer un polygone englobant pour la suppression du
contenu non-ligne, et elle est incapable de reconnaître conjointement les
lignes et les régions. En changeant la couche de sortie du réseau neuronal pour
effectuer une classification de pixels multi-étiquettes, la nouvelle méthode
est capable de détecter à la fois les lignes et les régions simultanément. Les
nouvelles classes de marqueurs dans la carte de sortie indiquent quelle
extrémite d'un ligne est le début et quelle fin permet de déterminer
l'orientation de la ligne. En outre, une nouvelle méthode de post-traitement a
été proposée pour extraire les lignes de base de la sortie de la carte de
pixels brute du réseau de segmentation sémantique et pour calculer un polygone
englobant. Enfin, le réseau de segmentation sémantique a été changé en un
réseau de type ReNet plus efficace en mémoire qui utilise des couches LSTM
balayées verticalement et horizontalement au lieu de piles profondes de couches
convolutionnelles pour obtenir de grands champs récepteurs.

Cette deuxième méthode a été évaluée à la fois sur l'ensemble de données
arabes, une nouvelle version de l'ensemble de données cBAD et un certain nombre
d'ensembles de données latines plus petites. Les résultats étaient comparables
à la pointe de la technologie pour la segmentation des régions et des lignes de
texte, avec une certaine amélioration par rapport à la méthode précédente dans
les résultats sur l'ensemble de données arabes.

Enfin, nous avons proposé une méthode simple pour la détection du système
d'écriture et de I'emphase dans le texte lignes. Ce système est utile pour le
traitement des textes et documents multilingues où l'emphase, c'est-à-dire le
texte en italique, en gras, etc. est utilisée pour le balisage sémantique, tel
qu'il se produit fréquemment dans les dictionnaires.

La méthode profite de l'alignement implicite fourni par le réseau de
transcription de texte formé avec la fonction de coût CTC.  Bien qu'il ne soit
pas garanti que les activations pour un caractère particulier soient proches de
son emplacement dans la ligne, les capacités limitées de modélisation à longue
distance d'un réseau LSTM font qu'il le place presque toujours correctement.
En entrainant un réseau de transcription de texte à produire une séquence de
codes d'identification au lieu de caractères réels, nous pouvons diviser une
ligne en bandes appartenant à un seul système d'écriture. Ces bandes peuvent
ensuite être traitées par des modèles de reconnaissance spécifiques à
l'écriture. Une propriété intéressante de notre approche est que le système
peut être entraîné et que les données de son apprentissage peuvent être
dérivées automatiquement des données d'apprentissage existantes pour les
modèles de transcription.

\section{La transcription et l'alignement}

\subsection{Le logiciel ROC Kraken}

Le Kraken est un logiciel de ROC modulaire et open source conçu pour être
particulièrement utile pour la rétro-numérisation dans les sciences humaines.
Outre les méthodes de pointe pour la transcription et l'analyse de la mise en
page, il comprend un certain nombre d'autres fonctionnalités qui le rendent
intéressant pour les chercheurs en humanités.

Un grand soin a été apporté à son développement pour réduire les hypothèses
implicites sur le fonctionnement du texte et pour rendre ses limitations
explicites. Il a été étendu depuis ses origines en tant que bifurcation du
système OCRopus avec un support Unicode complet de droite à gauche,
bidirectionnel et vertical de l'écriture, la détection des scripts et la
reconnaissance multigraphique. Une interface JSON simple permettant la
configuration d'un mappage entre les sorties de modèles numériques et les
séquences de points de code Unicode et vice versa. Ce mécanisme est
particulièrement utile pour les écritures logographiques de grande dimension
telles que le système d'écriture chinois car il permet la décomposition d'un
point de code Unicode représentant un seul groupe de graphèmes en ses
composants logiques dans la sortie du réseau neuronal.

Comme le Kraken est conçu pour être facilement intégré dans d'autres
applications, il offre à la fois une API simple et un système de sérialisation
flexible grâce à des templates. Des templates pour un certain nombre de formats
tels que ALTO, hOCR, et TEI sont fournis par défaut. Les modules de traitement
sont accessibles à la fois par l'API et par la ligne de commande qui permet la
substitution flexible de blocs fonctionnels ou l'utilisation de sous-systèmes
pour compléter ses propres méthodes.

Dans le cas où des défauts raisonnables sont souhaitables mais peuvent être
désavantageux dans les cas marginaux, ils peuvent généralement être désactivés
ou adaptés. Les exemples vont du traitement textuel tel que la prise en charge
de texte bidirectionnel\footnote{L'algorithme Unicode BiDi a des cas où un
balisage explicite de la directionalité peut être requis.} et de la
normalisation du texte au changement des architectures et des paramètres
d'entraînement des réseaux neuronaux artificiels employés dans la segmentation
et la transcription des pages.

Le module de transcription fonctionne comme un classificateur de séquences sans
segmentation, utilisant un réseau neuronal artificiel pour mapper une image
d'une seule ligne de texte en une séquence d'étiquettes qui sont ensuite
mappées en points de code Unicode.  L'ARN utilisé par défaut est un CNN-RNN
hybride entraîné avec la fonction de coût CTC. Un langage simple de
spécification de réseau permet d'adapter le réseau à des tâches spécifiques.
Les précisions des caractères pour un certain nombre de scripts différents
utilisant ce classificateur sont indiquées dans le tableau~\ref{tab:acc}.

La segmentation des pages est assurée par le système de segmentation des
régions et des lignes décrit ci-dessus. Comme d'autres parties du logiciel, il
est hautement configurable et permet la détection de régions et de lignes de
texte arbitraires avec suffisamment de données d'entraînement. Les données
d'entraînement peuvent être fournies dans un certain nombre de formats de
fichiers standard tels que ALTO et PageXML ou via une simple API.

\subsection{L'alignement des Caractères}

Une tâche d'un certain intérêt paléographique est l'alignement automatique de
la transcription du texte avec les glyphes respectifs dans une image. Bien que
cela puisse être fait naïvement avec une approche de segmentation des
caractères similaire aux anciens logiciels de ROC, nous avons évalué une
méthode qui utilise l'alignement implicite de la fonction de coût CTC pour
localiser les graphèmes dans une image, à partir d'une transcription
diplomatique, et nous l'avons comparée à un système SIFT-flow. La méthode est
destinée à fonctionner sur les manuscrits de la mer Morte, des manuscrits très
fragmentaires écrits principalement en hébreu.

Dans un premier stade, les manuscrits hébraïques fragmentaires sont segmentés à
l'aide d'un modèle de segmentation des pages spécifiquement entraîné pour ce
matériel. Les transcriptions diplomatiques par ligne de la base de données QWB
sont ensuite mises en correspondance avec la sortie du segmenteur afin de créer
des données d'entraînement pour un modèle de transcription de manière
semi-automatique.  Environ 2500 lignes provenant de 440 fragments ont ensuite
été utilisées pour faire une apprentissage par transfer d'un nouveau modèle de
transcription à partir d'un modèle de transcription de manuscrit hébraïque
médiéval existant.  Comme les données d'apprentissage varient énormément en
termes de style, les caractères individuels sont souvent gravement dégradés, et
le modèle est entraîné à surajuster sévèrement, le CER est assez élevé avec
environ 30\% sur l'ensemble de validation.

Les activations de ce modèle surajusté sont utilisées pour déterminer les
positions des caractères sur le matériel dans l'ensemble d'entraînage créé
semi-automatiquement. Lorsqu'il est évalué vis-à-vis des positions de glyphes
annotées par l'homme, le système place le caractère le plus proche de la
position réelle 90,3\% du temps avec une IoU moyenne de 0,81, surpassant
significativement la méthode SIFT-flow même lorsqu'il l'ancre avec les
positions brutes des caractères ROC.

\section{eScriptorium}

eScriptorium est une plateforme d'analyse et d'annotation de documents open
source. Elle cherche à combiner des techniques de calcul avec des outils
numériques manuels pour la transcription et l'annotation approfondie de textes
et d'images aux niveaux paléographique, philologique et linguistique. Il
s'adresse aux chercheurs en sciences humaines, mais aussi aux bibliothécaires
et archivistes, aux étudiants, aux informaticiens et au grand public. Issu du
projet Scripta, qui cherche à faciliter l'étude de l'écriture sous toutes ses
formes au fil de l'histoire, ses principes de base sont la transparence, la
flexibilité et l'indépendance de la langue et du système d'écriture.

Ce dernier point est particulièrement important car la gamme des langues et des
systèmes d'écriture étudiés dans le cadre de Scripta est énorme, couvrant le
Proche-Orient ancien, l'Iran et l'Asie centrale, l'Inde, l'Asie du Sud-Est et
de l'Est, ainsi que l'Occident classique et médiéval. Par conséquent, comme
pour le Kraken, un effort concerté a été fait pour réduire les hypothèses sur
le fonctionnement du texte.

eScriptorium utilise le Kraken pour ses besoins en vision par ordinateur.
Ainsi, la construction du pipeline ROC est reflétée dans l'interface
d'eScriptorium, avec une approche par étapes de l'importation des données, de
la segmentation des pages (automatique ou manuelle), de la transcription
(automatique ou manuelle), de l'annotation et de l'exportation.

La nécessité de s'adapter à une grande variété de systèmes d'écriture, en
particulier la volonté de pouvoir traiter des écritures rares et historiques,
impose à eScriptorium certaines restrictions de conception qui vont au-delà des
mesures prises pour rendre les méthodes computationnelles de Kraken
polyvalentes. Par définition, les langages rares manquent de grands ensembles
de données préexistants qui peuvent être utilisés pour lancer le processus de
ROC.  Par conséquent, l'annotation et la vérification manuelles de la
segmentation et de la transcription ne peuvent pas être une simple réflexion
après coup, mais doivent être considérées comme une partie fondamentale de
l'interface, à la fois pour permettre un travail pratique avec les plus petits
ensembles de données qui ne peuvent pas encore être traités avec les méthodes
automatiques mises en œuvre et pour aider au démarrage efficace du traitement
automatique.

Elle empêche également l'utilisation de techniques courantes pour augmenter la
généralisation et la charge de formation des méthodes automatiques telles que
les modèles de langage statistique et les modèles généralisés pour des tâches
comme la segmentation des pages. Les modèles linguistiques puissants pour les
langues à faibles ressources telles que le vietnamien ancien sont tout aussi
irréalistes qu'un segmenteur de pages capable d'extraire avec précision des
lignes d'inscriptions chinoises, de manuscrits arabes, d'incunables et de
journaux avec un seul modèle RNA. Par conséquent, la plate-forme est conçue
pour permettre un apprentissage et un réapprentissage fréquents grâce à des
inventaires de modèles, des interfaces intermédiaires pour l'importation et
l'exportation de données, et des rapports d'évaluation prospectifs à grain fin
comme ceux qui existent déjà dans le Kraken.

Un dernier aspect renforçant ces contraintes de conception dans la plate-forme
provient non pas du matériel source mais du type de travail effectué sur
celui-ci. Les chercheurs en humanités effectuent un large éventail de
recherches en utilisant un grand nombre de paradigmes différents sur le
matériel textuel. Ce pluralisme méthodologique se traduit par des conventions
de transcription différentes, même sur du matériel dans la même langue, en
fonction des préférences particulières du chercheur et de son domaine. Il
existe donc un besoin fondamental de s'adapter aux différentes normes et de les
rendre visibles aux autres, en particulier dans le contexte des systèmes
d'intelligence artificielle qui ne sont, après tout, que de puissants outils
d'inférence statistique. Les systèmes ouverts peuvent aider à communiquer les
normes et les hypothèses de ces procédures, mais il n'en reste pas moins que
pour un simple utilisateur de sciences humaines peu familier avec la
terminologie de l'informatique, celles-ci sont cachées dans une boîte noire
magique (et vice versa).  Les ontologies peuvent principalement combler ce
fossé, mais elles sont complexes à mettre en place et à entretenir et se
heurtent souvent à la nature ad hoc de la recherche. Notre meilleure option
reste de suivre les standards lorsqu'ils existent, d'offrir des interfaces pour
prendre et apporter des données et des artefacts depuis et vers l'eScriptorium,
et d'accepter qu'il est très peu probable qu'un seul outil soit à la fois
pratique et universel.

\section{Conclusions et Perspectives}

En conclusion, nous avons présenté dans cette thèse un travail qui représente
un pas en avant vers la rétro-numérisation pratique des documents en écriture
arabe et des documents historiques et non-latins en général. La segmentation
des pages et la transcription sont maintenant en principe capables de numériser
n'importe quel document en caractères arabes, mais surtout l'inclusion de ces
méthodes dans un système de ROC de bas niveau et un ERV de haut niveau, qui
sont tous deux totalement ouverts à l'adaptation, la réutilisation et le
partage, rendent l'utilisation de ces outils dans les projets de sciences
humaines numériques, petits et grands, beaucoup plus attrayante.

Il est clair qu'un travail substantiel reste à faire. Nous avons étudié les
exigences générales d'un système de ROC à usage général en écriture arabe et
validé l'état de l'art au début de la thèse dans deux études. Bien que les
questions les plus urgentes, l'analyse de la mise en page, des méthodes de
transcription plus puissantes et de meilleurs outils pour la création, la
conservation et la diffusion des données, aient été résolues dans une large
mesure, toutes les tâches ne sont pas actuellement résolues de manière
satisfaisante.

La tâche la plus urgente pour la ROC des manuscrits arabes est la détermination
de l'ordre de lecture. Comme décrit ci-dessus, la recherche sur ce sujet est
rare et les méthodes existantes sont des heuristiques artisanales incapables de
traiter la structure souvent complexe des manuscrits historiques. Alors que les
ensembles de données sont inexistants ou implicitement cachés dans des
ensembles de données pour d'autres tâches, les capacités de l'intelligence
artificielle pour le raisonnement spatial avec un grand nombre d'objets ont
augmenté ces dernières années avec le développement des réseaux neuronaux de
graphes.

Bien que les systèmes de ROC de pointe sont capables de réaliser des exploits
impressionnants même sur des documents très dégradés et atypiques, avec des
exigences en matière de données d'entraînement plus ergonomiques que jamais, le
repérage des données d'entraînement est toujours la tâche la plus longue des
techniques modernes de vision par ordinateur. Alors que nous pouvons maintenant
utiliser efficacement l'apprentissage par transfert pour adapter les modèles
existants à de nouveaux documents avec des quantités minimales de données, des
méthodes d'adaptation de domaine plus avancées offrent de grandes promesses
pour rendre plus de documents accessibles sans intervention humaine.

Le développement d'eScriptorium va sûrement se poursuivre et intégrer les
progrès des méthodes automatiques dans la mesure où il aide la recherche en
sciences humaines. Les pistes non explorées dans cette thèse comprennent les
opérations non textuelles, telles que diverses tâches de classification
d'images, la datation, le regroupement de documents similaires ou la détection
de la réutilisation de textes.

Enfin, même avec la disponibilité d'outils de vision par ordinateur puissants
et ouverts, le paysage des ensembles de données reste fracturé. Alors que les
chercheurs reconnaissent plus que jamais l'importance du partage des données
pour faire progresser non seulement les sciences humaines mais aussi la
recherche informatique, le moyen préféré pour y parvenir reste le dépôt github
profane avec un fichier README non descriptif. Une combinaison d'ERV conscients
de l'importance de métadonnées appropriées, d'un apport élargi de la pratique
archivistique dans la recherche scientifique, et de l'utilisation
d'infrastructures de données de recherche ouvertes comme c'est déjà le cas dans
d'autres disciplines scientifiques, a le potentiel d'améliorer considérablement
cet état de fait dans les prochaines années.

\end{french}


\cleardoublepage
\chapter{Technical Overview of the Kraken Software}
\chaptermark{Technical Overview of Kraken}
\label{app:kraken}

This appendix is a technical summary of the Kraken software in its current
state. It is valid for the version deposited for this thesis in the Zenodo
research data repository and assigned the DOI \emph{10.5281/zenodo.4498925}.

This document is located in between the low-level descriptions of the methods
and algorithms employed in chapters~\ref{ch:hip}, \ref{ch:icfhr}, and
\ref{ch:multi} and the high-level conceptual overview of the software intended
for humanists in chapter~\ref{ch:kraken}. The majority of the text is derived
from the technical end-user documentation available on the Kraken web site.

\section{Command Line Interface}

The principal way to interact with Kraken for most users is through the command
line interface (CLI). For practical purposes the CLI is split into two
principal parts, the \emph{kraken} command for all tasks related to inference, i.e.
recognition, and the \emph{ketos} command for tasks related to the training
and evaluation of segmentation and transcription models.

\subsection{Inference}

The \emph{kraken} command exposes each processing step of the OCR process as
a separate \emph{subcommand} which operates on a number of inputs to produce
specific output files. In concordance with the linear workflow structure of
OCR, these subcommands can be chained to perform multiple processing steps at
the same time.

The general invocation of the command is thus:

\begin{minted}{shell}
$ kraken -i inp_1 outp_1 -i inp_2 outp_2 ... subcmd_1 subcmd_2 ... subcmd_n
\end{minted}

Input files can be in different formats and defined in different ways. The
above syntax is the most direct: each input file is directly mapped to an
output file. As this syntax is too verbose for more than a few files and does
not allow the definition of multiple outputs for a single input file, as is
desirable for multi-page TIFF or PDF files, batch input are handled in two
different ways. The first allows the use of shell or glob patterns to match
multiple input files and append a specific suffix to each output files:

\begin{minted}{shell}
$ kraken -I glob_pattern -o suffix ...
\end{minted}

For example:

\begin{minted}{shell}
$ kraken -I '*.png'  -o '.xml' ...
\end{minted}

The second enables the splitting of a single input file into multiple output
files with dynamically created suffixes through a format string:

\begin{minted}{shell}
$ kraken -I glob_pattern -p format_string ...
\end{minted}

For example:

\begin{minted}{shell}
$ kraken -I '*.pdf' -p '{src}_{idx:06d}.xml' ...
\end{minted}

splitting each input file into output files starting with the original file
name followed by a page index.

A variety of input file formats are supported, both for reasons of convience
and because each processing steps expect different input data. As OCR is the
conversion of image data into machine-encoded text, the expected default is
unsurprisingly plain image files:

\begin{minted}{shell}
$ kraken -i image_1.png output_1 -i image_2.png output_2 ...
\end{minted}

More complex data can be fed into kraken with files in the ALTO and PageXML
formats. These are, for example, useful to only perform transcription on
already pre-segmented images. Each subcommand will automatically retrieve the
necessary information, i.e. executing the layout analysis subcommand on an ALTO
file will cause Kraken to only load the image file defined therein. Input
formats can be switched with the \emph{-f} switch, e.g.:

\begin{minted}{shell}
$ kraken -i alto.xml output.xml -f alto subcommand_1 subcommand_2 ...
\end{minted}

A special case are multi-page inputs. These can also be selected with the
appropriate \emph{-f} option, currently \emph{-f pdf} for both PDF and
multi-page TIFF files, but as they do not contain parseable structural and
content information only image data is extracted. Valid values for the format
option are currently \emph{alto}, \emph{page}, \emph{pdf}, \emph{image}, and
\emph{xml} (to automatically select the appropriate parser for each XML input
file).

\subsubsection{Binarization}

Binarization is no longer mandatory with the new segmenter but the original
OCRopus binarization algorithm is still available through the \emph{binarize}
subcommand.

\begin{minted}{shell}
$ kraken -i ... binarize
\end{minted}

\subsubsection{Layout Analysis}

Layout analysis is accessed with the \emph{segment} subcommand. Two
segmenters are implemented in Kraken, the legacy non-trainable segmenter
producing bounding box data and the new trainable segmenter that uses the
baseline and bounding polygon paradigm. In addition to extracting text lines,
the latter is also able to detect regions (both textual and non-textual) and
assign classes to text lines if trained with the appropriate training data. We
will only explain the use of the new segmenter here.

The segmenters can be selected with a subcommand option:

\begin{minted}{shell}
$ kraken -i ... segment -x # legacy segmenter
$ kraken -i ... segment -bl # baseline segmenter
\end{minted}

When the baseline segmenter is selected a default model trained on modern Latin
manuscripts will be used. This simple model only detects lines and a basic text
region. Other segmentation models can be supplied with the \emph{-i} option:

\begin{minted}{shell}
$ kraken -i ... segment -bl -i model_1.mlmodel
\end{minted}

It is also possible to run multiple segmentation models at the same time over
an image and obtain a joint segmentation:

\begin{minted}{shell}
$ kraken -i ... segment -bl -i line_seg.mlmodel -i region_seg.mlmodel
\end{minted}

This functionality can for example be used to combine the output of a
segmentation model that only produces regions with one that only detects text
lines. It is important to note that no filtering is performed on the output,
i.e. when combining multiple line-detecting segmentation models the output will
contain "duplicate", largely overlapping lines. Apart from the convenience of
merging multiple region and line segmentation automatically, performing joint
segmentation in Kraken also allows the segmenter to use additional region
information for bounding polygon calculation which generally improves polygon
accuracy, especially on lines close to the boundary of the writing surface.

The segmenter not only finds lines and regions but also imparts a reading order
on them using a basic heuristic. As Kraken does not know the principal text
direction of the document it can be supplied through an option
\emph{-{}-text-direction}: 

\begin{minted}{shell}
$ kraken -i ... segment --text-direction horizontal-lr # horizontal lines
                                                       # left before right lines
$ kraken -i ... segment --text-direction horizontal-rl # horizontal lines
                                                       # right before left lines
$ kraken -i ... segment --text-direction vertical-lr # vertical lines
                                                     # left before right lines
$ kraken -i ... segment --text-direction vertical-rl # vertical lines
                                                     # right before left lines
\end{minted}

This text direction is unrelated to the direction of the writing system in a
line and only determines the inter-line and column order. Taking a parallel
English and Arabic text as an example, it is possible that lines are read
top-to-bottom, left column before right column (the page is typeset
left-to-right, i.e. like a Latin-script document) or that lines are read
top-to-bottom, right column before left column (the page is typeset
right-to-left, i.e. like an Arabic document). The options for vertical lines
behave correspondingly.

In some cases it is desirable to mask out parts of the input image which are
known not to contain any lines or regions. Mask images have to be the same
shape as the input image. Black pixels in the mask image will be ignored be the
segmenter:

\begin{minted}{shell}
$ kraken -i ... segment -m mask.png ...
\end{minted}

The output of the segmenter is a JSON file containing the verbatim data
structure returned by the internal segmentation method of Kraken:

\begin{minted}[escapeinside=||]{json}
{"text_direction": "horizontal-lr",
 "type": "baselines",
 "lines": [{"script": "default", "baseline": [[877, 281], [1893, 318]],
                                 "boundary": [[877, 281], |\dots| [881, 325]]},
           {"script": "default", "baseline": [[1224, 552], [1351, 500]],
                                 "boundary": [[1224, 552], |\dots| , [1231, 555]]},
           |\dots|],
 "region": {"text": [[[500, 128], |\dots| [200, 325], [|\dots|],
            "illustrations": |\dots|},
 "script_detection": true
}
\end{minted}

\subsubsection{Transcription}

Transcription requires a color, grey-scale, or binarized image, a page
segmentation for said image, and a model file containing a transcription model.
The first two can either originate from earlier subcommands or directly from
an XML file. Model files are defined through the \emph{-m} option on the
\emph{ocr} subcommand:

\begin{minted}{shell}
$ kraken -i ... ocr -m trans.mlmodel
\end{minted}

The \emph{ocr} subcommand is multi-model capable, i.e. it is possible to
selectively apply transcription models on parts of the provided text lines.
Originally intended for multigraphic transcription (see
chapter~\ref{ch:multi}), this selection can be made for arbitrary criteria,
such as different hands, languages, or typefaces. The assignment of
transcription models to text lines works through line types which are part of
the segmentation parsed either from an XML file (see the examples in the Kraken
git repository for exact attributes used) or the output of an appropriately
trained layout analysis model (the value of the \emph{script} field in the
example segmentation above). It is therefore a simple mapping based on
classifications performed beforehand. The general syntax for this mapping is:

\begin{minted}[breaklines]{shell}
$ kraken -i ... ocr -m type_1:m_1 -m type_2:m_2 -m type_3:m_3
\end{minted}

Two special keywords exist for types and models. The \emph{default} identifier
is a catch-all and applies the specified model on every identifier that does
not have a model assigned explicitly. The \emph{ignore} model value causes
Kraken to ignore text lines with this identifier and silently drop them from
the output. If no \emph{default} model is defined, unassigned types will cause
Kraken to abort processing with an error message. An example for transcribing
all lines except those assigned the \emph{notes} type:

\begin{minted}{shell}
$ kraken -i ... ocr -m default:defmodel.mlmodel -m notes:ignore
\end{minted}

The default output of the \emph{ocr} subcommand is a plain text file with the
text in a line corresponding to the respective line in the segmentation. As
this output lacks metadata, such as line, word, and character locations, links
to image files, and utilized transcription models, enriched XML output formats
can be selected with options on the subcommand:

\begin{minted}{shell}
$ kraken -i ... ocr ... -t # text output
$ kraken -i ... ocr ... -h # hOCR output
$ kraken -i ... ocr ... -a # ALTO output 
$ kraken -i ... ocr ... -y # abbyyXML output
$ kraken -i ... ocr ... -x # PageXML output
\end{minted}

Output is serialized \emph{de novo}, i.e. even if an input file was already an
XML file in ALTO or PageXML output is not "inserted" into the input but the
segmentation and transcription are used to produce an entirely new file which
can lack information contained in the input file.

It is also possible to use the transcription functionality without a
segmentation through the \emph{-{}-no-segmentation} switch. In this case, each
input image is treated as one whole line instead of a document page containing
multiple text lines. 

\subsection{Training}

Training functionality is provided through subcommands of the \emph{ketos}
command line tool. There are three principal commands: \emph{train} and
\emph{test} for training and evaluating transcription models and
\emph{segtrain} for training layout analysis models.

While basic tooling for training data creation for transcription models was
included in the past, these are only compatible with the legacy bounding box
segmenter. For the annotation and transcription of baselines, regions, and
text external tools like eScriptorium or Aletheia that can either export data
in ALTO 4.2 and  PageXML format or have tight Kraken integration are the
preferred option.

Therefore, both transcription and layout analysis are trained primarily through
datasets contained in ALTO or PageXML files. Legacy formats, line images and
text files for transcription and JSON files containing line coordinate lists,
are still supported but do not offer the full range of functionality.

\subsection{Transcription Training and Evaluation}

Training a transcription model from a collection of PageXML or ALTO files
containing the necessary annotation (baselines, bounding polygons, and text)
can be done in two ways, from scratch or based on an existing model. The latter
is useful when a model for similar documents, such as a similar typeface or
hand, already exists. In this case, transfer learning to the new data can
reduce the training requirements substantially.

We will start with the simple case of training a model from scratch:

\begin{minted}{shell}
$ ketos train -f xml *.xml
[1.8139] alphabet mismatch: chars in training set only: ... (not included in accuracy test during training)
Initializing model ✓
stage 1/∞  [###################] 1163/1163 Accuracy report (1) 0.1844 10092 8231
stage 1/∞  [###################] 1163/1163 Accuracy report (1) 0.1844 10092 8231
stage 2/∞  [###################] 1163/1163 Accuracy report (2) 0.2335 10092 7736
stage 3/∞  [###################] 1163/1163 Accuracy report (3) 0.3242 10092 6820
stage 4/∞  [###################] 1163/1163 Accuracy report (6) 0.4006 10092 6049
...
\end{minted}

This command automatically parses the XML files in either of the supported
formats, loads the images, splits off ten per cent of the training data as a
validation set, and commences training. Training is divided into epochs, with
an evaluation automatically performed on the validation set after each line in
the training set has been seen at least once by the network. 

The warning about an alphabet mismatch is the result of the training dataset
containing characters that are not in the validation set. The network is not
evaluated against these characters but still learns how to recognize them. This
is usually the case with small datasets and rare characters. A corresponding
warning if a character is in the validation set but not in the training set
can also be printed.

Depending on the speed of the computer and the size of the data, training can
take a substantial amount of time. Per default training stops automatically as
soon as the character accuracy (the first number in the accuracy report in the
output above) on the validation set does not improve above a certain threshold
for a number of epochs. This approach, called early stopping, uses default
parameters that might not be appropriate for all datasets. For very small
datasets of only a few dozen lines the default number of epochs before aborting
(five) might be too low while very large datasets without much variation can
cause the model to overfit between evaluation runs. To adjust these parameters
a couple of options are available:

\begin{minted}{shell}
$ ketos train ... -F 0.5 # evaluates after half the training set
$ ketos train ... --lag 10 # waits 10 epochs for any improvement
$ ketos train ... --min-delta 0.001 # lowers improvement threshold to 0.1%
\end{minted}

Instead of a random split into training and validation set that changes with
each training run, it is also possible to force a fixed split to ensure
reproducibility acros runs. The most explicit way is through manifest files
that each contain the path to one XML file per line:

\begin{minted}{shell}
$ ketos train -f xml -t train.lst -e val.lst
\end{minted}

Transfer learning an existing model works similarly to training from scratch
but takes an existing model in addition to the training data:

\begin{minted}{shell}
$ ketos train -f xml -i model.mlmodel *.xml
[0.8616] alphabet mismatch {'~', '»', '8', '9', 'ـ'}
Network codec not compatible with training set
[0.8620] Training data and model codec alphabets mismatch: {'ٓA'}
\end{minted}

If the characters in the training set differ from the existing possible outputs
of the network, an error will be raised. As the transfer learning process
initially changes the internal structure of the model in a way that makes it
"forget" some of the already learned information, this is a basic safety
precaution. Two modes for adapting the model to the new alphabet: \emph{add}
and \emph{both}. \emph{add} resizes the model to be able to output all the
characters in the training set without removing any existing characters.
\emph{both} will make the resulting model an exact match with the training set
by removing both unused characters and adding new ones.

\begin{minted}{shell}
$ ketos train -f xml --resize add -i model.mlmodel *.xml
...
[0.8737] Resizing codec to include 1 new code points
[0.8874] Resizing last layer in network to 52 outputs
...
$ ketos train -f xml --resize both -i model.mlmodel *.xml
...
[0.7857] Resizing network or given codec to 49 code sequences
[0.8344] Deleting 2 output classes from network (46 retained)
...
\end{minted}

In this example 1 character was added for a network that is able to
recognize 52 different characters after sufficient additional training. It is
important to remember that in \emph{add} mode the model will first lose some
accuracy for characters it has already learned through the resizing process,
a deterioration that is worse for large changes, but also unlearn already
learnt characters that are not in the training set during training. This
initial deterioration is also true in \emph{both} mode but not the gradual
unlearning as all possible output characters are contained in the targeted
training data.

The command line interface for training also exposes various hyperparameters
such as model architecture, learning rate, optimizers, weight decay, etc. The
model architecture can be changed through VGSL (see section~\ref{vgsl}), while
various other parameters are set with options, such as \emph{-{}-optimizer},
\emph{-{}-lrate}, and \emph{-{}-weight-decay} (see the help message for valid
values). 

More command line options for various text normalizations, custom codecs,
recalculation of bounding polygons, caching of training data, etc. exist. These
are documented in the subcommand's help message and the full Kraken
documentation.

Lastly, it is possible to substantially accelerate training with CUDA
acceleration. This requires a properly configured graphics card (GPU) with
sufficient memory to place the model to be trained. As transcription models are
fairly small, all but the smallest GPUs are sufficient for this purpose. CUDA
acceleration is activated by selecting a GPU with the \emph{-{}-device} option:

\begin{minted}{shell}
$ ketos segtrain --device cuda ...
\end{minted}

After a model has been trained an in-depth analysis against a separate test
dataset is often performed. More detailled than the simple character accuracy
output during training and a better estimation of real world accuracy when
multiple models have been trained on the same training data, these reports
contain per-script accuracy rates and confusion matrices that can also
pin-point weaknesses of the transcription model:

\begin{minted}{shell}
$ ketos test -m best.mlmodel -f xml *.xml
Evaluating $model
Evaluating  [###################] 100%
=== report best.mlmodel ===

7012 Characters
6022 Errors
14.12%       Accuracy

2    Insertions
5226 Deletions 
794  Substitutions

Count Missed   %Right
1567  575    63.31%  Common
5230  5230   0.00%   Latin
215   215    0.00%   Inherited

Errors       Correct-Generated
773  { A } - {  }
536  { c } - {  }
328  { e } - {  }
274  { d } - {  }
...
\end{minted}

The report start off with an overall accuracy, followed by the absolute number
of errors and per-script\footnote{Scripts are determined according to Unicode
script property linked to ISO 15924 script codes which vary widely in
granularity. Script identifiers are defined for variant forms of the Latin
script such as Fraktur but only one identifier exist for Arabic and its derived
scripts.} accuracy rates. The remainder of the report contains the confusion
table sorted by frequency.

\subsubsection{Layout Analysis Training}

Training of layout analysis models is very similar to training of transcription
models, just with a different subcommand:

\begin{minted}{shell}
$ ketos segtrain -f xml *.xml
Creating model ...
Training line types:
  $pac  3  6539
  $not  4  202
  $par  5  14803
Training region types:
  $tip	6  829
  text	7  8
stage 1/50  [###################] 46/46 Accuracy report (1) mean_iu: 0.0309 freq_iu: 0.0975 mean_acc: 0.0309 accuracy: 0.0309
stage 2/50  [############-------] 16/46 00:05:11
\end{minted}

Instead of stopping automatically after a period of accuracy stagnation,
\emph{segtrain} stops after fifty epochs per default. This is chiefly because
the pixel accuracy rates are not directly linked to the actual baseline and
region detection quality.

As can be seen in the above example, the model is trained per default on all
the baseline types and regions in the training dataset (the object counts for
each type are listed after the type).  There are multiple options that control
this behavior. It is possible to suppress either baselines or regions
completely:

\begin{minted}{shell}
$ ketos segtrain --supress-baselines ...
$ ketos segtrain --suppress-regions ...
\end{minted}

More fine-grained controls allow the merging and suppression of specific types
with whitelists:

\begin{minted}{shell}
$ ketos segtrain --valid-regions reg_1 --valid-region reg_2 ...
$ ketos segtrain --valid_baselines type_1 --valid_baselines type2 ...
$ ketos segtrain --merge-baselines $par:$not
$ ketos segtrain --merge-regions $text:$tip
\end{minted}

Both can be combined. The region/baseline whitelists are processed before
merging, so it is necessary to whitelist even regions/baselines that are merged
into others with the \emph{-{}-merge-*} options.

Like transcription models, layout analysis models can be transfer learned to a
new dataset. The same two modes \emph{both} and \emph{add} exist. In contrast
to transcription models, adaptation does not directly affect the accuracy of
other types, although transfer learning in \emph{add} mode will still slowly
unlearn types not in the new training data:

\begin{minted}{shell}
$ ketos segtrain --resize add -f xml *.xml
$ ketos segtrain --resize both -f xml *.xml
\end{minted}

Likewise explicit splits between training and evaluation set can be provided:

\begin{minted}{shell}
$ ketos segtrain -f xml -t train.lst -e val.lst
\end{minted}

In the same vein, hyperparameters and GPU acceleration can be set through
identical options.

\subsubsection{VGSL}
\label{vgsl}
Kraken implements a dialect of the Variable-size Graph Specification Language
(VGSL), enabling the specification of different network architectures for image
processing purposes using a short definition string.

A VGSL specification consists of an input block, one or more layers, and an
output block. For example a grayscale line transcription network consisting of
two convolutional layers (ReLU activation) with 32/64 3$\times$3 filters,
followed2 $\times$2 maxpooling after each layer, and a final bidirectional LSTM
layer and 1D dropout regularization:

\begin{minted}{shell}
[1,48,0,1 Cr3,3,32 Mp2,2 Cr3,3,64 Mp2,2 S1(1x0)1,3 Lbx100 Do O1c103]
\end{minted}

or a simple layout analysis model pixel labelling (4 classes) a 1200 pixel high
RGB color image with two convolutional layers and one ReNet-like block:

\begin{minted}{shell}
[1,1200,0,3 Cr3,3,64 Gn32 Cr3,3,128 Lby32 Lbx32 O2l4]
\end{minted}

The first block defines the input in order of (batch, heigh, width, channels)
with zero-valued dimensions being variable. Integer valued height or width
input specifications will result in the input images being automatically scaled
in either dimension. Mixed variable and fixed input sizes, e.g. a height set to
400 and a width set to 0, will result in a proportional scaling of the image.
The batch size is currently ignored in Kraken and can be set separately with
command line options. 

When channels are set to 1 grayscale or B/W inputs are expected, 3 expects RGB
color images. Higher values in combination with a height of 1 result in the
network being fed 1 pixel wide grayscale strips scaled to the size of the
channel dimension, i.e. an internal transposition of the height and channel
dimensions.

After the input, a number of processing layers are defined. Layers operate on
the channel dimension; this is intuitive for convolutional layers but a
recurrent layer performing sequence classification along the width axis on an
image of a particular height requires the height dimension to be moved to the
channel dimension, e.g.:

\begin{minted}{shell}
[1,48,0,1 S1(1x0)1,3 Lbx100 O1c103]
\end{minted}

or using the aforementioned alternative formulation performing the
transposition implicitly with the input definition:

\begin{minted}{shell}        
[1,1,0,48 Lbx100 O1c103]
\end{minted}

Finally an output definition \emph{O\dots} is appended. When training
transcription and segmentation models with the provided command line tools
these are derived automatically from the training data based on the number of
different code points or baseline and region types.

The two principal layer types available in VGSL are LSTM and GRU layers:

\begin{minted}{text}
L(f|r|b)(x|y)[s]<n> LSTM cell with n outputs.
G(f|r|b)(x|y)[s]<n> GRU cell with n outputs.
  f runs the LSTM/GRU forward only.
  r runs the LSTM/GRU reversed only.
  b runs the LSTM/GRU bidirectionally.
  x runs the LSTM/GRU in the x-dimension.
  y runs the LSTM/GRU in the y-dimension.
  s (optional) summarizes the output in the requested dimension,
     outputting only the final step, collapsing the dimension to a
     single element.
\end{minted}

and convolutional layers:

\begin{minted}{text}
C(s|t|r|l|m)<y>,<x>,<d>[,<y_stride>,<x_stride>]
Convolves using a y,x window, with optional stride, valid padding, d outputs,
with configurable non-linearity.
(s|t|r|l|m) specifies the type of non-linearity:
s = sigmoid
t = tanh
r = relu
l = linear (i.e., None)
m = softmax
\end{minted}

Multiple auxiliary layers exist. The \emph{S} layer shuffles data between
dimensions and is most frequently used to collape any remaining y-height before
the recurrent layers in a transcription model:

\begin{minted}{text}
S<d>(<a>x<b>)<e>,<f> Splits one dimension, moves one part to another dimension.
Takes dimension d, reshapes it into a (a,b)-shaped tensor, distributing a into
dimension e and b into dimension f. Setting a or b to 0 auto-fills to the
correct value.
\end{minted}

Various regularization layers are implemented:

\begin{minted}{text}
Do<p>[,(1|2)] Inserts a 1D or 2D dropout layer with probability <p>. Defaults
to 1D dropout.
Gn<g> Inserts a group normalization layer with <g> groups.
\end{minted}

Maxpooling:

\begin{minted}{text}
Mp<y>,<x>[,<y_stride>,<x_stride>] Adds a maxpooling layer with kernel size
	(y,x) and optional stride (y_stride,x_stride)
\end{minted}

and finally output layers:

\begin{minted}{text}
O(0|1|2)(l|s|c)<d> Adds an output layer for scalar, 1D, or 2D heatmap output
with d classes.
(l|s|c) select both non-linearity and loss function:
l = sigmoid (binary crossentropy)
s = softmax (crossentropy)
c = softmax (CTC loss)
\end{minted}

As mentioned above output layers are added automatically by command line tools,
so it is only necessary to create them when using the API.



\subsection{Model repository}

Kraken incorporates a simple model repository that stores layout analysis and
transcription models with basic metadata in the
Zenodo\footnote{\url{http://zenodo.org}} research data repository. Models made
available through the repository are public and can either be retrieved with
Kraken's command line tools, through the Zenodo website, or its web API.
Because of limitations of the Zenodo platform, publishing of models is
currently not completely automated and requires manual approval of each
submitted model by an administrator. Therefore publishing models is not
instantaneous until the necessary changes to Zenodo's API are made to enable
automatic approval.

Models in the repository are interacted with through DOI permanent identifiers.
As these are globally unique, unalterable, and resolvable to the object in the
repository; they can be used to reference a particular model precisely, e.g. in
publications. To retrieve the list of models in the repository:

\begin{minted}{shell}
$ kraken list
Retrieving model list .✓
10.5281/zenodo.2577813 (pytorch) - A generalized model for English printed text
....
\end{minted}

The get more details on the exact type of data, character accuracy, etc. of the
model one can also retrieve the metadata record of a single model with its DOI:

\begin{minted}{shell}
$ kraken show 10.5281/zenodo.2577813
name: 10.5281/zenodo.2577813

A generalized model for English printed text

This model has been trained on a large corpus of modern printed English text
augmented with ~10000 lines of historical printed documents.
scripts: Latn
alphabet: !"#$%&'()+,-./0123456789:;<=>?@ABCDEFGHIJKLMNOPQRSTUVWXYZ[]`abcdefghijklmno
          pqrstuvwxyz{} SPACE
accuracy: 99.95%
license: Apache-2.0
author(s): Kiessling, Benjamin
date: 2019-02-26
\end{minted}

The record contains a natural language description of the models, describing
usually the amount and type of training data used, and additional information
like ISO 15924 script identifiers, code points in the model codec, character
accuracy on the test set, and the authors's names.

To actually download a model, one simply executes:

\begin{minted}{shell}
$ kraken get 10.5281/zenodo.2577813
Retrieving model ...
Model name: en_best.mlmodel
\end{minted}

Models are placed per default in the local user configuration directory which
is often \emph{.config/kraken} but can vary between operating systems. The
Kraken subcommands search for models automatically in this directory so they
can be used directly with:

\begin{minted}{shell}
$ kraken ... segment -i seg.mlmodel
$ kraken ... ocr -m en_best.mlmodel
\end{minted}

no matter where those commands are executed.

Models are published with the \emph{ketos publish} command. As it accesses the
Zenodo API it requires an access token which can be generated in the web
interface of the platform by any account holder. The \emph{publish} subcommand
asks for a number of values to fill the metadata record and uploads the record
and model to Zenodo:

\begin{minted}{shell}
$ ketos publish arabic.mlmodel
Access token: $SUPER_SECRET
author: foobar
affiliation:
summary: this is a model for all arabic text ever written
accuracy on test set: 100.0
script: Arab
license: SISSL
Uploading .....
model PID: 10.5281/zenodo.2577814
\end{minted}

The record is created immediately and the PID is valid but an administrator has
to approve the record's accession to the OCR model group in Zenodo in order for
it to be discoverable with Kraken's command line tools.

\subsection{API}

A simple API is available for both training and inference. The principal
recognition tasks are encapsulated in single functions and a simple pipeline
for OCR is only a few lines of Python code:

\begin{minted}{python}
from PIL import Image

from kraken import blla, rpred
from kraken.lib import models, vgsl

seg_model = vgsl.TorchVGSLModel.load_model('la.mlmodel')
tr_model = models.load_any('tra.mlmodel')

im = Image.open('/path/to/image.png')
seg = blla.segment(im, model=tr_model)
for line in rpred.rpred(tr_model, im, seg):
    print(line.prediction)
\end{minted}

These lines load first import the PIL library for image handling, the
\emph{blla} Kraken module for (baseline) segmentation, the \emph{rpred} module
for text transcription, and the \emph{models} and \emph{vgsl} modules for model
loading. Afterwards, the respective layout analysis and transcription models,
and an input image are loaded. The difference between the two model loading
function is that transcription models are wrapped in a slim abstraction layer
while segmentation models use the raw VGSL interface directly.

Next, we perform segmentation on the previously loaded image file and
transcribe each found line with the transcription model. The transcription
function \emph{rpred} does not only return text but an object containing also
character bounding polygons and confidences. \emph{rpred} is a simplified
transcription method for single model transcription; multi-model functionality
capable of transcribing typed lines with multiple models is available through
the more advanced \emph{mm\_rpred} function (see the full API documentation for
further details).

Training is more complicated but a basic training run with the default
parameters is just a few lines of code as well:

\begin{minted}{python}
from kraken.lib import xml
from kraken.lib.train import KrakenTrainer

training_files = ['a.xml', 'b.xml', 'c.xml']
eval_files = ['d.xml', 'e.xml']

# callback called after each iteration 
def step_callback(*args):
    return lambda: print('.')

# function to print the validation results after each epoch
def print_transcription_eval(epoch, accuracy, chars, error, **kwargs):
    print(f'Accuracy report {epoch} {accuracy} {chars} {error}')

# create a transcription model trainer
t_trainer = KrakenTrainer.recognition_train_gen(progress_callback=step_callback,
                                                output='model.mlmodel',
                                                training_data=training_files,
                                                evaluation_data=eval_files,
                                                format_type='xml')

# executing the transcription trainer.
t_trainer.run(print_transcription_eval)

# function to print the segmentation validation results after each epoch
def print_seg_eval(epoch, accuracy, mean_acc, mean_iu, freq_iu, **kwargs):
    print(f'Accuracy report ({epoch}) mean_iu: {mean_iu}')

# create a layout analysis model trainer
la_trainer = KrakenTrainer.segmentation_train_gen(progress_callback=step_callback,
                                                  output='seg.mlmodel',
                                                  training_data=training_files,
                                                  evaluation_data=eval_files,
                                                  format_type='xml')

# executing the transcription trainer.
la_trainer.run(print_seg_eval)

# retrieve of epoch of best validation error
print(f'best transcription model error: {t_trainer.stopper.best_epoch}')
print(f'best segmentation model error: {la_trainer.stopper.best_epoch}')
\end{minted}

The basic principle is simple. A \emph{KrakenTrainer} object is created through
the constructors for transcription or layout analysis training. These
constructors accept the arguments and options already known from the command
line (see the API documentation for further details) and a callback that is
executed each time after a sample has been ran through the neural network.
Training runs are initiated by calling the \emph{run} method on the object with
another callback that is executed after the evaluation at the end of each
epoch.  \emph{run} blocks and automatically returns once training is finished
according to the stop parameters chosen, per default early stopping for
transcription and a fixed number of epochs for layout analysis models.

Most options available on the command line are available on the respective API
functions. A complete overview can be found on the Kraken website and in the
Zenodo deposit mentioned in the introduction with example scripts showing
low-level use contained in the contrib directory of the source code.

\end{appendices}

\end{document}
